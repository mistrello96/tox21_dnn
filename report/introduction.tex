\pagenumbering{arabic}
\begin{abstract}
Lo scopo di questo lavoro è quello di sviluppare una rete neurale per classificare delle molecole in base al loro potenziale di interferie con i \textit{pathways} di segnalazione dei recettori nucleari; il dataset utilizzato è \href{http://bioinf.jku.at/research/DeepTox/tox21.html}{Tox21}.
In primo luogo, si è riconosciuta la natura multi-label del problema (\textit{i.e.}, 12 label differenti di cui predire il valore); inoltre, si è mostrato come il dataset è estremamente sbilanciato per ogni label evidenzando l'impossibilità di bilanciarlo.
Si sono effettuate delle operazioni di \textit{preprocessing} sul dataset, quali l'imputazione dei valori mancanti di alcune label, la rimozione di feature con varianza nulla e dei \textit{sample} considerati outlier.
Al fine di individuare un classificatore ottimale per svolgere la \textit{challenge}, si è brevemente indagato come la profondità della rete neurale influisse nelle performance del modello, i risultati indicano che una profondità maggiore non porta a migliorare le performance, portando a prediligere una rete neurale con solo 3 layer nascosti la cui ampiezza viene dimezzata.
Successivamente, è stato effettuato un processo di ottimizzazione degli iperparametri per indiviuarne la configurazione ottima e stabilire quale tra quattro possibili input della rete utilizzare; i risultati riportano che una strategia di feature selection basata sulla correlazione sia la strategia ottima.
Infine, si è addestrato il modello con la configurazione di iperparametri ottima e con input il dataset dopo l'operazione di feature selection, confrontandone le performance con gli altri modelli presenti in letteratura sfruttando un insieme di dati mai utilizzati durante le fasi precedenti del lavoro.
I risultati mostrano che il nostro modello si posiziona a “metà classifica” con performance parzialmente peggiori rispetto a varie tecniche \textit{ensamble}.
\end{abstract}

\section{Introduzione}
Il problema affrontato in questo lavoro è quello dello sviluppo di un classificatore per i dataset Tox21 \cite{challenge_site}, una \textit{challenge} proposta dal \textit{U.S. Department of Health and Human Services} che consiste nel predire il risultato di dodici differenti test tossicologici eseguiti su delle molecole a partire da un insieme di descrittori della struttura chimica dei composti proposti.
L'obiettivo della cosiddetta \textit{Grand Challenge}, affrontata in questo lavoro, è quello di predire il risultato di tutti e dodici i test contemporaneamente; i test possono essere catalogati in due macro-gruppi: i primi $7$, identificati dal prefisso \textit{NR}, afferiscono all'area dei test relativi ai \textit{pathways} di segnalazione dei recettori nucleari (\textit{nuclear receptor effects}); i restanti $5$, identificati dal prefisso \textit{SR}, sono relativi alla risposta cellulare agli stress (\textit{stress responde effects}).
La richiesta risulta essere particolarmente complessa, in quanto per ogni classe solo un piccolo sottogruppo di \textit{sample} risulta positivi al test, portando il problema ad essere fortemente sbilanciato e senza possibilità di bilanciamento, vista la natura multi-label della predizione.
In questo lavoro, al fine di individuare un classificatore ottimale per svolgere la \textit{challenge}, sono stati posti i seguenti obiettivi: (i) confrontare diverse topologie di rete; (ii) individuare la tecnica di feature reduction ottimale; (iii) determinare tramite un processo di ottimizzazione la configurazione migliore degli iperparametri dei modelli ; (iv) analizzare le performance del modello individuato.
\pagenumbering{arabic}
\begin{abstract}
Lo scopo di questo lavoro è quello di sviluppare una rete neurale per la classificazione di composti chimici in base alla possibile tossicità rispetto ad alcuni \textit{pathways} cellulari; il dataset utilizzato per la prova è \href{http://bioinf.jku.at/research/DeepTox/tox21.html}{Tox21}.
In primo luogo, è stata evidenziata la natura multi-label del problema (\textit{i.e.}, 12 label differenti di cui predire il valore); inoltre, è stato mostrato come il dataset sia estremamente sbilanciato per ogni label, evidenziando l'impossibilità di bilanciarlo a causa della natura delle etichette.
Dopo aver effettuato operazioni di \textit{preprocessing}, è stato brevemente indagato come la profondità della rete neurale influisse sulle performance del modello; i risultati mostrano come una profondità maggiore non garantisca prestazioni migliori, portando a scegliere per il lavoro un rete con tre layer nascosti.
Successivamente è stato effettuato un processo di ottimizzazione degli iperparametri, al fine di individuarne la configurazione ottima e stabilire quale tra quattro possibili input preprocessati della rete utilizzare.
Infine è stato addestrato il modello con la configurazione di iperparametri e input ottimale, confrontandone le performance con altri modelli presenti in letteratura.
I risultati mostrano che l'approccio proposto si posiziona a “metà classifica”, con performance parzialmente peggiori rispetto a varie tecniche \textit{ensamble}.
\end{abstract}

\section{Introduzione}
Il problema affrontato in questo lavoro è quello dello sviluppo di un classificatore per i dataset Tox21 \cite{challenge_site}, una \textit{challenge} proposta dal \textit{U.S. Department of Health and Human Services} che consiste nel predire il risultato di dodici differenti test tossicologici eseguiti su delle molecole a partire da un insieme di descrittori della struttura chimica dei composti proposti.
L'obiettivo della cosiddetta \textit{Grand Challenge}, affrontata in questo lavoro, è quello di predire il risultato di tutti e dodici le etichette contemporaneamente. I test possono essere catalogati in due macro-gruppi: i primi $7$, identificati dal prefisso \textit{NR}, afferiscono all'area dei test relativi ai \textit{pathways} di segnalazione dei recettori nucleari (\textit{nuclear receptor effects}); i restanti $5$, identificati dal prefisso \textit{SR}, sono relativi alla risposta cellulare agli stress (\textit{stress responde effects}).
Analizzando i dati, si evince che il problema risulta essere fortemente sbilanciato e senza possibilità di bilanciamento, vista la natura multi-label della predizione.
Per l'individuazione di un classificatore ottimale al fine di svolgere la \textit{challenge}, sono stati posti i seguenti obiettivi: (i) confrontare diverse topologie di rete; (ii) individuare la tecnica di feature reduction ottimale; (iii) determinare tramite un processo di ottimizzazione la configurazione migliore di iperparametri per i modelli ; (iv) analizzare le performance del modello individuato.
\section{Conclusioni}
Nel lavoro proposto è stato dapprima esplorato il dataset, costatando la natura multi-label del problema e il forte sbilanciamento delle classi verso esempi con esito negativo. Dopo aver effettuato operazioni di pulizia e rimozione degli outlier, il dataset è stato utilizzato per addestrare alcune reti con iperparametri fissati, in modo da individuare se una topologia dalla profondità maggiore potesse produrre performance migliori. Constatato l'esito negativo dell'esperimento precedente, la topologia della rete è stata fissata, ed è stato eseguito un processo di ottimizzazione di alcuni iperparametri per ogni tecnica di feature reduction proposta, oltre che sull'intero dataset. I risultati hanno portato a selezionare la riduzione tramite correlazione, che è stata utilizzata in concomitanza con la configurazione di iperparametri ottima individuata dal processo di ottimizzazione per addestrare la rete conclusiva. Questo modello è stato utilizzato poi per le operazioni di predizione sui dati di test messi a disposizione per la \textit{challenge} e le performance sono state confrontate con diversi altri approcci presenti in letteratura; la tecnica proposta si posiziona al centro della classifica, con performance variegate a seconda della classe. Vista la natura del dominio analizzato, si rende necessario un confronto con un esperto del campo, in modo da stabilire con criterio il reale costo delle predizioni errate e selezionare così dei \textit{threshold} appropriati per il processo di inferenza.
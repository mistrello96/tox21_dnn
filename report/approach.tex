\section{Approccio e metodi utilizzati}
A seguito dell'analisi sulle feature, che ha evidenziato una correlazione tra un discreto numero di esse, sono state sviluppate e utilizzate tre differenti tecniche di feature reduction. 
La prima è una feature selection basata sulla correlazione; \textit{i.e.}, elimina quegli elementi che risultano correlati oltre una certa soglia con altri elementi. 
Ponendo questa soglia a $0.90$, la dimensione dell'input passa da $793$ a $414$ feature.
Le rimanenti due tecniche utilizzate, che afferiscono ai metodi di feature extraction, sono la Principal Component Analysis (PCA) e l'utilizzo di un autoencoder; in entrambi i casi, il numero di feature estratte è stato fissato a $100$.
Per quanto concerne la struttura dell'autoencoder, è stata utilizzata una topologia con 2 layer nascosti prima del \textit{bottleneck}, i layer sono composti da un numero di neuroni dimezzato progressivamente, i quali sfruttano una \texttt{relu} come funzione di attivazione. 
Al contrario, il layer di output utilizza una funzione lineare, in quanto i valori attesi sono numeri reali.\\
Al fine di individuare la migliore topologia da utilizzare per il successivo processo di ottimizzazione sono state poste a confronto tre reti dalla profondità crescente; la struttura di base si compone di $3$ layer nascosti con attivazione \texttt{relu}, dropout $0.2$ e normalizzazione dopo l'attivazione, con rispettivamente $512$, $256$ e $128$ neuroni. 
Ad essa, sono stati aggiunti in modo incrementale due layer con le medesime caratteristiche e con numero di neuroni dimezzato rispetto al livello precedente. %(\textit{i.e.}, 64 e 32 neuroni rispettivamente). 
In coda alle reti così create è stato posto un layer di output con $12$ neuroni attivati mediante una sigmoide, poiché le label attese assumono valori pari a 0 o 1; l'ottimizzatore utilizzato per il processo di training è stato \texttt{adam} (scelto per la maggiore efficienza e il \textit{learning rate} dinamico) e il parametro di peso della funzione di \textit{loss} è stato posto a $20$ (valore prossimo allo sbilanciamento complessivo tra le classi). 
Il valore appena citato influenza il peso che la funzione di loss associa ad un errore sulla classe positiva: questa funzione di \textit{loss} permette quindi di gestire problemi fortemente sbilanciati come nel caso preso in analisi.
Il fine di questa operazione è quello di determinare se una topologia più profonda della rete sia in grado di ottenere performance migliori a parità di valori degli iperparametri.

Una volta individuata la migliore tra le topologie proposte, è stato effettuato un processo di ottimizzazione bayesiana di alcuni iperparametri del modello. 
Questa tecnica, a differenza di approcci più \textit{na\"ive}, è \textit{sample efficient}, ovvero sfrutta al meglio il budget a disposizione per individuare la configurazione ottimale degli iperparametri. 
Nel processo sono stati inclusi: (i) dimensione del \textit{batch}; (ii) valore di dropout; (iii) peso della funzione di regolarizzazione all'interno della funzione obiettivo; (iv) funzione di attivazione dei layer nascosti; (v) peso della funzione di loss per la gestione dello sbilanciamento tra le classe.
Vista la presenza di variabili sia continue che categoriche, il modello surrogato selezionato è stato quello delle \textit{random forest}, maggiormente adatto alla gestione di valori non continui. 
Il budget messo a disposizione del processo è stato fissato a 120 valutazioni, di cui il 12,5\% è stato utilizzato come \textit{initial design} del modello surrogato, mediante un campionamento dello spazio Latin Hypercube Sampling (LHS). 
Il valore ottimizzato dal processo è stato il valore medio delle AUC (Area Under Curve) sulle $12$ classi in \textit{3-fold cross validation}; la funzione di acquisizione utilizzata è stata la Lower Confidence Bound (LCB).
Il processo di Automated Machine Learning (AutoML) è stato effettuato considerando separatamente come input della rete sia il dataset ottenuto dopo la fase di \textit{preprocessing}, sia le tre tecniche di feature reduction (applicate al dataset dopo le operazioni di pulizia).
Così facendo, è stato possibile individuare la configurazione di iperparametri ottimi per i vari possibili input e confrontare le performance dei modelli associati ad essi.

Avendo stabilito la combinazione tra feature reduction e iperparametri in grado di massimizzare la AUC media, è stato quindi analizzato il modello derivante più nel dettaglio.
Il classificatore così ottenuto è stato utilizzato come metodo di predizione per i dati di test e le performance ottenute sono state confrontate con quelle dei modelli proposti in letteratura.
Oltre a considerare le AUC inerenti alle singole classi e media, sono state analizzate le curve Receiver Operating Characteristic (ROC) riguardanti le singole classi e si è studiato come all'evolvere della percentuale di True Positive (\%TP) desiderata, evolvono le misure di precision e recall inerenti alle singole classi.
\section{Approccio e metodi utilizzati}

This is the central and most important section of the report. Its objective must be to show, with linearity and clarity, the steps that have led to the definition of a decision model. The description of the working hypotheses, confirmed or denied, can be found in this section together with the description of the subsequent refining processes of the models. Comparisons between different models (e.g. heuristics vs. optimal models) in terms of quality of solutions, their explainability and execution times are welcome. 

Do not attempt to describe all the code in the system, and do not include large pieces of code in this section, use pseudo-code where necessary. Complete source code should be provided separately (in Appendixes, as separated material or as a link to an on-line repo). Instead pick out and describe just the pieces of code which, for example:
\begin{itemize}
\item are especially critical to the operation of the system;
\item you feel might be of particular interest to the reader for some reason;
\item  illustrate a non-standard or innovative way of implementing an algorithm, data
structure, etc..
\end{itemize}

You should also mention any unforeseen problems you encountered when implementing the
system and how and to what extent you overcame them. Common problems are:
 difficulties involving existing software.
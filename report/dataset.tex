\section{Dataset utilizzato}
Il dataset utilizzato per il progetto è stato prelevato dal sito \cite{dataset_site}, che mette a disposizione i dati di train ($12060$ \textit{sample}) e di test ($647$ \textit{sample}) per lo svolgimento della \textit{challenge} associata al problema.
Ogni elemento del dataset rappresenta un differente composto chimico ed è caratterizzato da un insieme di $801$ feature dense e oltre $200000$ feature sparse. Considerata la complessità computazionale aggiuntiva derivante dall'utilizzo di queste ultime feature a fronte della minima quantità di informazione apportata, è stato deciso di non considerare la porzione sparsa della matrice, come consigliato dal sito citato in precedenza.
Ad ogni \textit{sample} è associato un vettore binario di etichette contenente $12$ elementi.
Ognuno di essi corrisponde al risultato di una differente analisi tossicologica a cui la molecola è stata sottoposta. Le analisi effettuate possono essere unite in due macro-gruppi: le prime $7$, identificate dal prefisso NR (\textit{nuclear receptor effects}), afferiscono all'area dei test relativi ai \textit{pathways} di segnalazione dei recettori nucleari; le restanti $5$, identificate dal prefisso SR (\textit{stress responde effects}), sono relative alla risposta cellulare agli stress.

L'analisi della matrice delle etichette ha mostrato una discreta presenza di valori \texttt{NaN} (35\% nel train set, 10\% nel test set); non essendo presente alcuna motivazione di questi valori nella descrizione del dataset, è stato supposto che alcune analisi siano state tralasciate per quelle molecole che i ricercatore hanno ritenuto certamente non tossiche in determinati \textit{pathways}. Per questo motivo, tutti i valori \texttt{NaN} sono stati sostituiti con $0$, rappresentando quindi un esito del test tossicologico negativo per il composto chimico.\\
\begin{figure}[!ht]
	\centering
	\subfloat{\includegraphics[width=.2\textwidth]{../images/pdf/hist-NRAhR}}\quad
	\subfloat{\includegraphics[width=.2\textwidth]{../images/pdf/hist-NRAR}}\quad
	\subfloat{\includegraphics[width=.2\textwidth]{../images/pdf/hist-NRARLBD}}\quad
	\subfloat{\includegraphics[width=.2\textwidth]{../images/pdf/hist-NRAromatase}}\\	\subfloat{\includegraphics[width=.2\textwidth]{../images/pdf/hist-NRER}}\quad
	\subfloat{\includegraphics[width=.2\textwidth]{../images/pdf/hist-NRERLBD}}\quad
	\subfloat{\includegraphics[width=.2\textwidth]{../images/pdf/hist-NRPPARgamma}}\quad
	\subfloat{\includegraphics[width=.2\textwidth]{../images/pdf/hist-SRARE}}\\
	\subfloat{\includegraphics[width=.2\textwidth]{../images/pdf/hist-SRATAD5}}\quad
	\subfloat{\includegraphics[width=.2\textwidth]{../images/pdf/hist-SRHSE}}\quad	\subfloat{\includegraphics[width=.2\textwidth]{../images/pdf/hist-SRMMP}}\quad
	\subfloat{\includegraphics[width=.2\textwidth]{../images/pdf/hist-SRp53}}
	\caption{Distribuzione delle etichette dei differenti test tossicologici.}
	\label{fig:class_distribution}
\end{figure}
L'attenzione è stata poi spostata sull'analisi della distribuzione delle etichette, riportata in Figura \ref{fig:class_distribution}, che mostra un forte sbilanciamento tra il numero di 0 ed il numero di 1 per ogni classe. Come già accennato in precedenza, il problema trattato ricade sotto la categoria dei task di classificazione multi-label. Per questo motivo, anche se le classi si sono rivelate sbilanciate, non è stato possibile effettuare operazioni di \textit{over/under-sampling}, in quanto la generazione di un nuovo \textit{sample} sintetico (o la rimozione di un record nel caso complementare) per una classe avrebbe influenzato lo sbilanciamento delle classi rimanenti. \\
L'analisi della correlazione tra le etichette è stata indagata mediante una rappresentazione \textit{heatmap} della matrice di correlazione, riportata in Figura \ref{fig:labelscorrmatrixheatmap}. I valori di correlazione risultano essere molto bassi in media, salvo per due coppie di label, dove troviamo valori nell'intorno di $0.50$. Analizzando la semantica delle etichette, è emerso che le due coppie di label correlate in modo significativo riguardano l'effetto tossico in un caso sull'intero recettore, nell'altro solo su una porzione di esso.
\begin{figure}
	\centering
	\includegraphics[width=0.7\linewidth]{../images/pdf/labels_corr_matrix_heatmap}
	\caption{Heatmap della matrice di correlazione delle feature. Solo due coppie di feature risultano correlate in modo significativo tra loro.}
	\label{fig:labelscorrmatrixheatmap}
\end{figure}

Per prima cosa è stata verificata la presenza di \textit{missing value} e il tipo dei dati contenuti nella matrice delle feature. L'analisi non ha evidenziato la presenza di alcun valore mancante e i dati si sono rivelati tutti di tipo numerico continuo. \todo{Non c'erano anche degli interi? DP}
Successivamente, sono state eseguite alcune operazioni di pulizia del dataset: rimuovendo $7$ feature con varianza pari a $0$ (quindi erano prive di qualsivoglia informazione) e $425$ \textit{sample} considerati come oultiler; sono stati identificati come outlier quei \textit{sample} che avessero almeno $\frac{1}{4}$ dei valori che eccedessero $3$ volte lo scarto interquantile delle relative feature.
L'analisi della correlazione tra feature e label non ha mostrato la presenza di feature altamente correlate alle label, con un valore di correlazione massimo pari a $0.35$; le feature maggiormente correlate alle diverse etichette si sono inoltre rivelate essere tutte differenti (tranne in un caso), rafforzando l'ipotesi di un'assenza di uno stretto legame tra le etichette e un'unica feature.
In Figura \ref{fig:distributionhighcorr}, è riportata un'analisi di come le varie feature/etichette sono risultate altamente correlate ($> 0.90$) con altre feature/etichette. È possibile vedere come la larga maggioranza degli elementi sia correlato con un basso numero di feature (si noti la scala logaritmica sull'asse \textit{x}).
Si trova, poi, un insieme di feature correlate con un numero di elementi compreso tra $5$ e $30$, mentre una decina di feature risultano correlate a più di 40 elementi. Questo suggerisce che sia opportuno eseguire un'operazione di feature reduction prima di eseguire l'addestramento del modello, in modo da ridurre la ridondanza di informazione \todo[inline]{Se proprio vuoi dire qualcosa metterei una cosa tipo: permettendo una maggio re capacità di generalizzazione al modello. DP} e velocizzare così sia il processo di train sia quello di inferenza. \todo[inline]{velocizzare il processo di inferenza e train mi sembra un'affermazion forte. Soprattutto per l'inferenza. Il maggior tempo di train dipende di più dai layer nascosti imho DP}
\begin{figure}
	\centering
	\includegraphics[width=0.7\linewidth]{../images/pdf/distribution_high_corr}
	\caption{Distribuzione di feature/label altamente correlate. È stata utilizzata la scala logaritmica per l'asse x.}
	\label{fig:distributionhighcorr}
\end{figure}
Infine, i dati sono stati  standardizzati con media nulla e deviazione standard unitaria, in modo da non creare squilibri, in termini di magnitudine dell'input dei modelli neurali presentati nelle prossime sezioni; ciò è stato fatto in quanto feature con scale differenti possono portare instabilità nella rete, causando la creazione di pesi troppo elevati che rendono le predizioni fortemente suscettibili a piccole variazioni dell'input.